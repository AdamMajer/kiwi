\index{KIWI images!vmx|(}
\chapter{VMX image - Virtual Disks}
\label{chapter:vmx}
\minitoc

A VMX image is a virtual disk image for use in full virtualisation
systems like QEMU or VMware. The image represents a file which
includes partition data and bootloader information. The size of
this virtual disk can be influenced by either the <size> element
in your config.xml file or by the parameter $--$bootvm-disksize

\section{Building the suse-vm-guest example}

The latest example provided with kiwi is based on openSUSE 11.0 and makes use
of the base pattern. The operating system is stored on a
standard ext3 filesystem.

\begin{Command}{12cm}
\begin{verbatim}
cd /usr/share/doc/packages/kiwi/examples
cd suse-11.0
kiwi --prepare ./suse-vm-guest \
     --root /tmp/myvm
\end{verbatim}
\end{Command}

\begin{Command}{12cm}
\begin{verbatim}
kiwi --create /tmp/myvm \
     --type vmx -d /tmp/myvm-result
\end{verbatim}
\end{Command}

\section{Using the image}

The generated virtual disk image serves as the harddisk of the selected
virtualisation system. The setup of the virtual hard disk differes from
the variety of the virtualisation systems. A very simply to use system
is the QEMU virtualisation software. To run your image in qemu call:

\begin{Command}{12cm}
\begin{verbatim}
cd /tmp/myvm-result
qemu suse-11.0-vm-guest.i686-1.1.2.raw -m 256
\end{verbatim}
\end{Command}

\section{Flavours}

Because there are many virtualisation systems available there are also
many virtual disk formats. The .raw format KIWI always creates has the
same structure as you can find on a real hard disk. For virtualisation
software it makes sense to have specific formats to increase the
I/O performance when reading or writing data onto the disk from within
the virtual system. If you want to tell KIWI to create an additional
disk format just extend the type information of the config.xml file
by a format attribute.

\begin{Command}{12cm}
\begin{verbatim}
<type ... format=''name''>vmx</type>
\end{verbatim}
\end{Command}

The following table shows a list of supported virtual disk formats

\begin{tabular}[h]{|p{2cm}|p{9cm}|}
\hline
\textbf{Name} & \textbf{Description} \\
\hline
vvfat       & Disk format DOS FAT32 \\
vpc         & Virtual PC read only disk  \\
bochs       & Disk format for Bochs emulator  \\
dmg         & Disk format for Mac OS X \\
cloop       & Compressed loop \\
vmdk        & Disk format for VMware \\
ovf         & Open Virtual Format requires VMwares ovftool \\
qcow2       & QEMU virtual disk format  \\
qcow        & QEMU virtual disk format  \\
cow         & QEMU virtual disk format \\
\hline
\end{tabular}

\subsection{VMware support}

VMware is a very popular and fast virtualisation platform which is
the reason why KIWI has special support for it. VMware requires a
so called guest configuration which includes information about what
hardware should make up the guest and how much ressources should be
provided to the guest. With KIWI you can provide the information
required to create a guest configuration as part of the config.xml
file. Additionally you can group special packages which you may only
need in this virtual environment.

\begin{Command}{13cm}
\begin{verbatim}
<packages type="vmware">
   <!-- packages you need in VMware only -->
</packages>
<vmwareconfig memory="512">
   <vmwaredisk controller="ide" id="0"/>
</vmwareconfig>
\end{verbatim}
\end{Command}

If this information is present KIWI will create a VMware guest
configuration with 512 MB of RAM and an IDE disk controller interface.
Additional information to setup the VMware guest machine properties are
explained in the \textbf{vmwareconfig} section. The written guest
configuration file can be easily loaded and changed by the native
graphics user provided with VMware. The KIWI VMware guest configuration
is stored in the file:

\begin{Command}{13cm}
\begin{verbatim}
/tmp/myvm-result/suse-11.0-vm-guest.i686-1.1.2.vmx
\end{verbatim}
\end{Command}

Together with the \textbf{format=''vmdk''} attribute KIWI creates
a VMware based image (.vmdk file) and the required VMware guest
configuration (.vmx)

You can also create an image for the Xen virtualization framework.  To do this,
you simply need to specify the 'xen' \textbf{bootprofile} and
the 'nex' \textbf{bootkernel} in your config.xml type section. Don't be
curious about the name 'nex' as bootkernel. The reason for this is just because the name of the boot profile is xen and the name of the kernel is also xen.
Therefore I simply decided to use the reverse name for it. Like VMware,
Xen has a configuration file as well. Refer to chapter
\ref{chapter:xen} (Xen image) for details.

\subsection{LVM support}

kiwi also provides support for LVM (Logical Volume Management). In this
mode the disk partition table will include one lvm partition and one
standard ext2 boot partition. kiwi creates the kiwiVG volume group and
adds logical volumes as they are needed and configured according to the
image type and filesystem. After boot of the system the user has full
control over the volume group and is free to change/resize/increas the
group and the volumes inside. Support for LVM has been added for all
image types which are disk based. This includes vmx,oem and usb.
In order to use LVM for the vmx type just add the \verb+--lvm+ option
as part of the kiwi create step or add the attribute \textbf{lvm=''true''}
as part of the \textbf{type} section in your config.xml file.

\begin{Command}{12cm}
\begin{verbatim}
kiwi --create /tmp/myvm --type vmx \
     -d /tmp/myvm-result --lvm
\end{verbatim}
\end{Command}

With the optional \textbf{lvmvolumes} section you can set one or
more top level directories into a separate volume. For example you
want to put /usr and /var into different volumes add the following
lvmvolumes section:

\begin{Command}{12cm}
\begin{verbatim}
<lvmvolumes>
   <volume name="usr" freespace="100M"/>
   <volume name="var"/>
</lvmvolumes>
\end{verbatim}
\end{Command}

The optional freespace attribute let you add additional free space
for that volume. If you don't set any freespace the volume will be
about 80\% to 90\% full.
